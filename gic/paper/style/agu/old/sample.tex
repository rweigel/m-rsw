
%%  SAMPLE.TEX -- August 6, 1996 version
%
%%  AGU's sample and template file

   % This document demonstrates how to use commands 
   % described in the aguguide.tex user manual.  Most 
   % instructions and explanations are commented out; 
   % thus they will not appear if you LaTeX this file 
   % and view it as a sample document.

   % You may use this file as a template for your
   % camera-ready LaTeX document by replacing the
   % sample text with your own text and deleting 
   % any extraneous commands.

   % Note that this sample was created using the jgrga.sty
   % and therefore the lines will break accurately when
   % that style file is used.  If one of the other style 
   % files is used in the \documentstyle command, the
   % linebreaks may occur in odd places.  Use \linebreak
   % or \\ commands or \- (optional hyphen) commands to 
   % correct this as needed.

%% ------------------------------------------------------ %%

   % The first item in a LaTeX file must be a \documentstyle
   % command to declare the overall style of the paper.  The
   % \documentstyle lines that are relevant for AGU's LaTeX 
   % style files are shown; all but one are commented out so 
   % that the file can be processed.

   % \documentstyle[12pt,agums]{article}
     % This substyle creates double-spaced wide-margined 
    % manuscripts.  Use it to create copy to submit 
        % to AGU editors for first-time review.

   % \documentstyle[agupp]{article}
      % This substyle creates two-column preprints 
   % (i.e., for distribution to colleagues; do not 
        % submit two-column manuscripts to AGU).

   % \documentstyle[rtjga]{article}
     % This substyle creates single-column camera-ready 
     % manuscripts for Russian translation journals.

   % \documentstyle[grlga]{article}
      % This substyle creates single-column camera-ready 
     % manuscripts for Geophysical Research Letters.

   % \documentstyle[radga]{article}
      % This substyle creates single-column camera-ready 
     % manuscripts for Radio Science.

   % \documentstyle[tecga]{article}
     % This substyle creates single-column camera-ready 
     % manuscripts for Tectonics.

\documentstyle[jgrga]{article}
      % This substyle creates single-column camera-ready 
     % manuscripts for three journals: the Journal of 
       % Geophysical Research, Paleoceanography, and 
        % Global Biogeochemical Cycles.

%% ------------------------------------------------------ %%
%%
%%  PREAMBLE
%%
%%
   % Running heads, manuscript dates, manuscript 
   % information, copyright information, author 
   % addresses and slug comments are placed in the 
   % preamble (before the \begin{document} command).  
   %
   % AGU's style files cause the actual text to be 
   % typeset after the reference section.  If you 
   % have no reference section, you must include a 
   % \forcesluginfo command or the preamble information
   % will not print.


%% ------------------------------------------------------ %%
%
%%  RUNNING HEADS
%
%% ------------------------------------------------------ %%


\lefthead{KAY ET AL.}
    % Type only the last name(s) of the author(s) in all
    % capital letters between the curly brackets in the 
    % \lefthead command.  If your manuscript has three 
        % or more authors, type only the last name of the 
   % first author followed by "ET AL."

\righthead{LATEX SAMPLE DOCUMENT FOR AGU MANUSCRIPTS}
        % Type a short version of the article title (or 
        % include the entire title, if it will fit) in 
 % all capital letters between the curly brackets 
       % in the \righthead command.  You may type a total 
     % of 60 characters (including letters, punctuation, 
    % and spaces) between the curly brackets in the 
        % \lefthead or \righthead commands (i.e., 10 
   % characters in one command and 50 in the other
 % is acceptable).


%% ------------------------------------------------------ %%
%
%%  MANUSCRIPT DATES
%
%% ------------------------------------------------------ %%

\received{January~3,~1996}
\revised{February~27,~1996}
\accepted{March~31,~1996}
   % Editors supply the received, revised, and 
    % accepted dates.  Type your manuscript dates 
  % between the curly brackets in the \received, 
 % \revised, and \accepted commands.  Use the 
   % date format shown above.

%% ------------------------------------------------------ %%
%
%%  MANUSCRIPT INFORMATION    (GRL authors may leave these 
%%                             commands blank)
%
%% ------------------------------------------------------ %%

\paperid{96JA00000}
        % Type your AGU paper number between the curly 
 % brackets in the \paperid command.  The first
  % two numbers indicate the year the manuscript
  % was received at AGU headquarters.  The two
    % letters identify the AGU journal.  The last 
  % five numbers indicate in what order the
       % manuscript was received at AGU headquarters.

\cpright{AGU}{1996}
   % \cpright{PD}{1996}
   % \cpright{Crown}{1996}
      % If your paper is AGU copyright, choose the 
   % "AGU" \cpright command.  If your paper is in 
 % the public domain, choose the "PD" \cpright 
  % command.  If your paper is Crown copyright, 
  % choose the "Crown" \cpright command.  If you 
 % are not sure which copyright to choose, please 
       % contact your production coordinator.  You must 
       % choose one of these copyright options.

\ccc{0148-0227/96/96JA-00000\$09.00}
    % Type your Copyright Clearance Center code between
     % the curly brackets in the \ccc command.  Crown
        % copyrights and manuscripts in the Public Domain
       % have no "\ccc{}" information.  Be sure to place
       % a backslash in front of the dollar sign or it
 % will be interpreted as a math command.

%% ------------------------------------------------------ %%
%
%%  AUTHOR ADDRESSES 
%
%% ------------------------------------------------------ %%

\authoraddr{J. Blythe, 
Observation Center for Prediction of 
Earthquakes and Volcanic Eruptions, 
Faculty of Science, Tohoku University, 
Sendai 980, Japan.}

\authoraddr{M. Chen and K. Jamison, 
U.S. Geological Survey, 
345 Middlefield Road, 
MS 977, Menlo Park, CA  94025.  
(e-mail: k\_jamison@gold.wr.usgs.gov)}

\authoraddr{W.~F. Johnson, G.~M. Kay, 
T. Ryan, J. Smith, L. Song, and S.~C. Wilson, 
Department of Geology and Geophysics,
Woods Hole Oceanographic Institution, 
Woods Hole, MA  02543.  (e-mail: 
wj@red.whoi.edu; gk@blue.whoi.edu; 
tr@blue.whoi.edu; js@red.whoi.edu; 
ls@red.whoi.edu; sw@red.whoi.edu)}

       % For author postal addresses, group authors 
   % by affiliation and list them in alphabetical 
 % order.  Names include first initial only, 
    % optional middle initial, and last name.  If 
  % available, include e-mail addresses as shown.
        %
        % E-mail addresses may be broken over a period 
 % or after the @ sign.  If your e-mail address 
 % includes an underscore, do not forget to 
     % place a backslash in front of it or it will 
  % be interpreted as a command (see e-mail in 
   % second \authoraddr for example).  You may add 
        % additional \authoraddr commands if necessary.
 % 
      % Note that author addresses include complete 
  % postal address information, while author 
     % affiliations include only department, 
        % institution, town, and state (if state is 
    % not included in the name of the institution)
        % or country (if not the United States).

%% ------------------------------------------------------ %%
%
%%  SLUG COMMENTS (optional)
%
%% ------------------------------------------------------ %%

\slugcomment{To appear in the {\it Journal 
of Geophysical Research}, 1996.}
 % Slug comments appear only in manuscript and 
  % preprint styles, not in the camera-ready styles.


%% ------------------------------------------------------ %%
%
%%  CITE COMMANDS WITH OLD VERSION OF LATEX 2.09
%
%% ------------------------------------------------------ %%


%%  update.tex -- August 6, 1996 version
%
%%  revisions by Amy Hendrickson, TeXnology Inc.

%   You may not need this file unless you have 
%   a very old version of LaTeX 2.09.
%
%   This file keeps old versions of LaTeX 2.09 
%   from becoming confused with \reset@font and 
%   thus crashing at every undefined label or 
%   citation.  
%
%   If you do need to include this file, 
%   type 
%%  update.tex -- August 6, 1996 version
%
%%  revisions by Amy Hendrickson, TeXnology Inc.

%   You may not need this file unless you have 
%   a very old version of LaTeX 2.09.
%
%   This file keeps old versions of LaTeX 2.09 
%   from becoming confused with \reset@font and 
%   thus crashing at every undefined label or 
%   citation.  
%
%   If you do need to include this file, 
%   type 
%%  update.tex -- August 6, 1996 version
%
%%  revisions by Amy Hendrickson, TeXnology Inc.

%   You may not need this file unless you have 
%   a very old version of LaTeX 2.09.
%
%   This file keeps old versions of LaTeX 2.09 
%   from becoming confused with \reset@font and 
%   thus crashing at every undefined label or 
%   citation.  
%
%   If you do need to include this file, 
%   type \input{update.tex} in front of your 
%   \begin{document} command in your *.tex
%   file and then continue as you would ordinarily.

\makeatletter
\let\reset@font\empty
\makeatother
 in front of your 
%   \begin{document} command in your *.tex
%   file and then continue as you would ordinarily.

\makeatletter
\let\reset@font\empty
\makeatother
 in front of your 
%   \begin{document} command in your *.tex
%   file and then continue as you would ordinarily.

\makeatletter
\let\reset@font\empty
\makeatother


   % If you are using a very old version of LaTeX 2.09
   % and include \cite commands, LaTeX may have trouble
   % with the \reset@font command.  If so, LaTeX will 
   % pause at every undefined citation the first time 
   % the input file is LaTeXed.  Authors may either press
   % [Enter] each time this occurs in their first LaTeX 
   % run, or they may type 
%%  update.tex -- August 6, 1996 version
%
%%  revisions by Amy Hendrickson, TeXnology Inc.

%   You may not need this file unless you have 
%   a very old version of LaTeX 2.09.
%
%   This file keeps old versions of LaTeX 2.09 
%   from becoming confused with \reset@font and 
%   thus crashing at every undefined label or 
%   citation.  
%
%   If you do need to include this file, 
%   type 
%%  update.tex -- August 6, 1996 version
%
%%  revisions by Amy Hendrickson, TeXnology Inc.

%   You may not need this file unless you have 
%   a very old version of LaTeX 2.09.
%
%   This file keeps old versions of LaTeX 2.09 
%   from becoming confused with \reset@font and 
%   thus crashing at every undefined label or 
%   citation.  
%
%   If you do need to include this file, 
%   type 
%%  update.tex -- August 6, 1996 version
%
%%  revisions by Amy Hendrickson, TeXnology Inc.

%   You may not need this file unless you have 
%   a very old version of LaTeX 2.09.
%
%   This file keeps old versions of LaTeX 2.09 
%   from becoming confused with \reset@font and 
%   thus crashing at every undefined label or 
%   citation.  
%
%   If you do need to include this file, 
%   type \input{update.tex} in front of your 
%   \begin{document} command in your *.tex
%   file and then continue as you would ordinarily.

\makeatletter
\let\reset@font\empty
\makeatother
 in front of your 
%   \begin{document} command in your *.tex
%   file and then continue as you would ordinarily.

\makeatletter
\let\reset@font\empty
\makeatother
 in front of your 
%   \begin{document} command in your *.tex
%   file and then continue as you would ordinarily.

\makeatletter
\let\reset@font\empty
\makeatother
 in front of 
   % the \begin{document} command.  The "update.tex" file 
   % is part of AGU's LaTeX style file package and is 
   % available on the kosmos.agu.org anonymous ftp site.

%% ------------------------------------------------------ %%
%
%%  NUMBERING YOUR SECTIONS
%
%% ------------------------------------------------------ %%

\setcounter{secnumdepth}{4}

   % If you want LaTeX to automatically number your 
   % sections, type \setcounter{secnumdepth}{4} in
   % the preamble of your input file, as shown. If 
   % you do not want section numbers, type
   % \setcounter{secnumdepth}{0}.
  
%% ------------------------------------------------------ %%
%%
%%  \BEGIN{DOCUMENT} and FRONT MATTER
%%
%%
   % The body of the paper starts with the \begin{document}
   % command, which is followed by the front matter (title,
   % author, affiliation, and abstract).

\begin{document}


%% ------------------------------------------------------ %%
%
%%  TITLE
%
%% ------------------------------------------------------ %%

\title{Sample document for authors using \LaTeX\ \\ 
to prepare AGU manuscripts}

 % Type the title of your manuscript between the 
        % curly brackets in a \title command.  Capitalize
       % only acronyms, first letter of the first word, 
       % first letter of proper nouns, first letter of 
        % the first word after colons, and first letter 
        % of the first word of a subtitle.  If the title 
       % exceeds one line, break it so that the first line
     % is longer than the second line; break the title 
        % before articles, prepositions, and conjunctions.  
  % To break the title, type a double backslash where
     % you want the break to occur, as shown above.


%% ------------------------------------------------------ %%
%
%%  AUTHOR NAMES, AFFILIATIONS, and ALTERNATE AFFILIATIONS
%
%% ------------------------------------------------------ %%

\author{Geoffrey M. Kay, 
 Karen Jamison,\altaffilmark{1}
 William F. Johnson, 
 Thomas Ryan,\altaffilmark{2} \\
 Sara C. Wilson, and Justin Smith}
%
\affil{Woods Hole Oceanographic Institution, 
 Woods Hole, Massachusetts}

\author{J. Blythe and M. Chen\altaffilmark{1}}
%
\affil{Observation Center for Prediction of 
 Earthquakes and Volcanic Eruptions, Faculty 
 of Science, Tohoku University, Sendai, Japan}

\author{L. Song}
%
\affil{Woods Hole Oceanographic Institution, 
 Woods Hole, Massachusetts}

\altaffiltext{1}{Now at U.S. Geological Survey, 
 Menlo Park, California.}

\altaffiltext{2}{Also at Observation Center for 
 Prediction of Earthquakes and Volcanic Eruptions, 
 Faculty of Science, Tohoku University, Sendai, Japan.}

%%%% Type author names between the curly brackets in 
   % the \author command(s).  Author names consist of
   % first name or initial, optional middle name or 
   % initial, and last name.  If it is necessary to 
   % break lines, make sure you break them between 
   % two author's names (using \linebreak commands).

%%%% Type affiliation information between the curly 
   % brackets in an \affil command that immediately 
   % follows each \author command.

%%%% If your authors have alternate affiliations, 
   % type the footnote number between the curly 
   % brackets in an \altaffilmark command and type 
   % the alternate affiliation information in 
   % \altaffiltext commands below the \author and 
   % \affil commands, as shown.  There is a separate
   % \altaffiltext for each alternate affiliation 
   % indicated above.  This information prints
   % as footnotes after the reference section in 
   % your camera-ready copy, and will be placed 
   % on the bottom left-hand corner of the first 
   % page of your published manuscript.

%% ------------------------------------------------------ %%
%
%%  FOOTNOTED AFFILIATIONS    (For more than three 
%%                             author/affiliation groups.)
%
%% ------------------------------------------------------ %%

   % Important:  If your manuscript contains more than three 
   % author and affiliation groupings, all of the affiliate 
   % information should be footnoted using \altaffilmark{} 
   % commands as shown.  Please include commas inside the 
   % footnote numbers.

   % \author{   S. Kurtz,\altaffilmark{1} 
   %              Evan R. King,\altaffilmark{1,2} 
   %            A. E. Simon,\altaffilmark{1,3} 
   %             R. J. Hall,\altaffilmark{4} 
   %                C. D. Qian,\altaffilmark{5} 
   %                Jason M. Albert,\altaffilmark{5}
   %            G. G. Kaye,\altaffilmark{6} 
   %                and Antony Ignatov\altaffilmark{7}     }
   % 
   % \altaffiltext{1}{Astronomy Department, 
   %                  University of California, Berkeley.}
   % \altaffiltext{2}{Also at Physics Department, 
   %                  University of California, Berkeley.}
   % \altaffiltext{3}{Now at National Center for Atmospheric
   %                  Research, Boulder, Colorado.} 
   % \altaffiltext{4}{Space Telescope Science Institute, 
   %                  Baltimore, Maryland}
   % \altaffiltext{5}{Woods Hole Oceanographic Institution, 
   %                  Woods Hole, Massachusetts}
   % \altaffiltext{6}{Space Telescope Science Institute,
   %                  Baltimore, Maryland}
   % \altaffiltext{7}{Department of Astronomy, California 
   %                  Institute of Technology, Pasadena.}


%% ------------------------------------------------------ %%
%
%%  ABSTRACT
%
%% ------------------------------------------------------ %%

\begin{abstract}
This tutorial includes codes and explanations which 
will not print in an ordinary \LaTeX\ document.  Also 
included are samples of marked-up equations, tables, 
and figure captions.  In order to use this tutorial 
you should view or print it using an editor or a word 
processing program.
\end{abstract}

     % The abstract consists of a one-paragraph summary
      % of your paper, 250 words or fewer.  Do not cite
       % references unless absolutely necessary.  If you
       % must cite, place the citation in square brackets
        % using italic type (this abstract citation format
        % is new).  Do not include displayed material.

%% ------------------------------------------------------ %%
%
%%  SECTION HEADS
%
%% ------------------------------------------------------ %%

%%%% LEVEL 1 HEADS
   %
   % Use \section{} commands to identify Level 1 heads; 
   % type the appropriate Level 1 head between the curly 
   % brackets, as shown.
   %
   % Capitalize the first letter of each word (except for 
   % prepositions, conjunctions, and articles that are 3 
   % letters or shorter).
   %
   % Do not hyphenate Level 1 heads.  To break lines, 
   % type \protect\\ where you want a break to occur.
   %
   % If you wish to number your sections, see the
   % "Numbering Sections" information in the preamble,
   % above.

% \section{Introduction}


%%%% LEVEL 2 HEADS
   % 
   % Use \subsection{} command to identify Level 2
   % heads; type the appropriate Level 2 head between 
   % the curly brackets, as shown.
   % 
   % Capitalize the first letter of each word (except for
   % prepositions, conjunctions, and articles that are 3
   % letters or shorter).
   %
   % Do not hyphenate Level 2 heads.  To break lines,
   % type \protect\\ where you want a break to occur.

% \subsection{This is a Level Two Head}


%%%% LEVEL 3 HEADS
   %
   % Use the command \subsubsection{} to identify
   % Level 3 heads; type the appropriate head between
   % the curly brackets, as shown.
   %
   % Capitalize only acronyms, the first letter of the
   % first word, first letter of proper nouns, and first
   % letter of the first word after colons.
   %
   % Hyphenation is permitted in Level 3 heads, if needed.
   %
   % You must include at least two Level 3 heads per
   % Level 2 head.

% \subsubsection{This is a level three head}


%%%% LEVEL 4 HEADS
   %
   % Use the command \subsubsubsection{} to identify
   % Level 4 heads; type the appropriate head between
   % the curly brackets, as shown.
   %
   % Capitalize only acronyms, the first letter of the
   % first word, first letter of proper nouns, and first
   % letter of the first word after colons.
   %
   % Hyphenation is permitted in Level 4 heads, if needed.
   %
   % A level 4 head cannot directly follow a Level
   % 3 head; there must be a least one sentence between
   % the two heads.

% \subsubsubsection{This is a level four head}


%% ------------------------------------------------------ %%
%%
%%  TEXT
%%
%%
   % The body of the paper must start with a 
   % \begin{article} command, and you must include 
   % an \end{article} command after the references 
   % section (otherwise your text will not print at 
   % the proper column width).

   % The first \section{} command is optional.  You may 
   % delete the command or you may replace the word 
   % "Introduction" with text appropriate to your article.

\begin{article}
\section{Introduction}

   A focal problem today in the dynamics of globular 
clusters is core collapse.  It has been predicted by 
theory for decades [\markcite{{Baker et al.,} 1977;}
\markcite{{\it Barbour and Major,} 1977}; 
\markcite{{\it Bell,} 1972}], but observation has been 
less alert to the phenomenon.  For many years the 
central brightness peak in M15 [\markcite{{\it Brosche 
and Sunderman,} 1977}; \markcite{{\it Yamazaki,} 1978}] 
seemed a unique anomaly.

\section{Ragged Right Level 1 Head Longer Than One 
Line With a Flush Left Wraparound}

Then \markcite{{\it Higgins} [1968]} suggested 
a central peak in NGC 6397, and limited photographic 
surveys of \markcite{{\it Degges and Smith} [1977]} and 
\markcite{{\it Golitsyn and Mokhov} [1978]} were 
conducted, including NGC 6624, whose sharp center had 
often been noted [e.g., \markcite{{\it Namias,} 1974]}.

\section{Ragged Right Level 1 Head Longer Than One Line 
With a Flush Left Wraparound Line and 
$\protect \boldmath Italic $ Text}
   All our observations were short direct exposures with CCDs.  
At Central Observatory we used a TI 500$\times$500 chip and a 
GEC 575$\times$385 on the 1-m Nickel reflector.  The only 
filter available at Central was red.  At CTIO we used a GEC 
575$\times$385 with $B, V,$ and $R$ filters and an RCA 
512$\times$320 with $U, B, V, R,$ and $I$ filters on the 
1.5-m reflector.  

   The CCD images are unfortunately not always suitable 
for very poor clusters or for clusters with large cores, 
as seen in section \ref{text}.  

A number of star count profiles [\markcite{{\it Kennealy 
and Caledonia,} 1979}] as well as photoelectric profiles 
[\markcite{{\it McMullen,} 1978}] were reviewed.  In a 
few cases we judged normality by eye estimates on one 
of the Sky Surveys. 

%% ------------------------------------------------------ %%
   %
   % The next section shows \subsection commands. 
   % If subsections, subsubsections or subsubsubsections
   % are included in your manuscript, AGU style requires 
   % authors to use at least two per section (this is 
   % standard etiquette for any outline).
   % 
   % The next section also shows displayed math 
   % environments (as described in the AGU guide), 
   % and several mathematical typesetting examples.
   %
%% ------------------------------------------------------ %%

\section{A Level One Head}\label{text}
   It has been realized that helicity amplitudes provide
a convenient means for diagram evaluations.  These
amplitude level techniques are particularly convenient
for calculations involving many diagrams, where the
usual trace techniques for the amplitude squared become
unwieldy.  Our calculations use helicity techniques
[\markcite{{\it Hagiwara and Zeppenfeld,} 1986}]; 
we summarize below.

\subsection{Formalism}
A three-level amplitude in $ e^+e^-$ collisions can be
expressed in terms of fermion strings of the form
\begin{equation}
   \bar v(p_2,\sigma_2)P_{-\tau}\not\!a_1\not\!a_2
   \cdots\not\!a_nu(p_1,\sigma_1)\;,
\end{equation}
where $p$ and $\sigma$ label the initial $e^{\pm}$
four momenta and helicities $(\sigma = \pm 1)$,
$\not\!a_i=a^\mu_i\gamma_\nu$, and 
$P_\tau=\frac{1}{2}(1+\tau\gamma_5)$ is a chirality 
projection operator $(\tau = \pm1)$.  The $a^\mu_i$ 
may be formed from particle four momenta, gauge boson
polarization vectors, or fermion strings with an 
uncontracted Lorentz index associated with final state 
fermions.

\subsection{A Level Two Head}
In the chiral representation the $\gamma$ matrices 
are expressed in terms of $2\times 2$ Pauli matrices 
$\sigma$ and the unit matrix 1 as

%% ------------------------------------------------------ %%
%
%%  EQUATIONS
%
%% ------------------------------------------------------ %%

%%%% Single-line equations are centered.  For multiline 
   % equations, break each line at a sign of operation 
   % (+, -, etc.) if possible, with the sign of operation 
   % on the new line.  Indent second and subsequent lines 
   % to align with the first character following the equal 
   % sign on the first line.  If the left side of the 
   % equation is longer than one-third of the line, 
   % subsequent lines should be indented the same amount 
   % as a text paragraph indent (0.38cm, 0.15").  Set the 
   % equation number flush right on the last line of the 
   % equation with at least 4 mm (5/16 inch, 1 pica) of 
   % space between the equation and the equation number;  
   % if there is not enough room, set the number flush 
   % right on a separate line immediately following the
   % equation.  If you have two equations with the same 
   % equation number, set them over three lines of type, 
   % with the equation number appearing flush right and 
   % in parentheses on the second line.  

%%%% To create multi-line equations, use the
   % \begin{eqnarray} and \end{eqnarray} environment
   % to break equations into two or more lines, as
   % demonstrated below.  Use an \hspace{} command to
   % insert horizontal space into your LaTeX document.
   % Place an appropriate unit of measure between the
   % curly braces; you may have to experiment to achieve
   % the correct amount of space.  LaTeX accepts several
   % units of measurement (cm, in, etc.); you may use
   % whichever unit you prefer.

\begin{eqnarray}
   x_{1} & = & (x - x_{0}) \cos \Theta + 
        (y - y_{0}) \sin \Theta \\ 
   y_{1} & = & -(x - x_{0}) \sin \Theta + 
        (y - y_{0}) \cos \Theta.
\end{eqnarray}

%%%% You may change equation numbering by resetting 
   % the equation counter or by explicitly numbering 
   % an equation.
   % 
   % To explicitly number an equation, type \eqnum{}
   % (with the desired number between the brackets) 
   % after the \begin{equation} or \begin{eqnarray}
   % command.  The \eqnum{} command will affect only
   % the equation it appears with; LaTeX will number
   % any equations appearing later in the manuscript 
   % according to the equation counter.
   %
   % To reset the equation counter, place the following 
   % command in front of your equation(s).

\setcounter{equation}{0}

   % Set the equation counter to 0 if the next 
   % number needed is 1, or set it to 7 if the 
   % next number needed is 8, etc.
   %
   % The \setcounter{equation} command does affect 
   % equations appearing later in the manuscript.

%%%% If you have a multi-line equation that needs only
   % one equation number, use a \nonumber command in 
   % front of the double backslashes (\\) as shown in
   % equations above and below.

%%%% To add letters after equation numbers, place your 
   % equation or eqnarray within a \begin{mathletters} 
   % and \end{mathletters} environment.  This environment
   % can enclose several equations:

\begin{mathletters}
   \begin{eqnarray}
      \gamma^\mu & = & \left(\begin{array}{cc} 0 & 
      \sigma^\mu_+ \\ 
      \sigma^\mu_- & 0 \end{array} \right), 
      \;\gamma = \left( \begin{array}{cc} 
      \! \! -1 & \! \! 0 \\ \! \! 0 & \! \! 1  
      \end{array} \right), \\ & & \nonumber \\
      \sigma^\mu_{\pm} & = & ({\bf 1} ,\pm \sigma),
   \end{eqnarray}
giving 
   \begin{eqnarray}
      \not\!a= \left(\begin{array}{cc}0 & 
      (\not\!a)_+\\(\not\!a)_- & 
      0\end{array}\right), 
      \;(\not\!a)_\pm=a_\mu\sigma^\mu_\pm\;,
   \end{eqnarray}
\end{mathletters}

\subsubsection{Level three heads capitalize only 
the first letter of the first word, indent one-em 
space, and end with a period}
The spinors are expressed in terms 
of two-component Weyl spinors as
\begin{equation}
\eqnum{17b}      %% <-- Note the use of \eqnum{}
   u=\left(\begin{array}{c}(u)_-\\(u)
   _+\end{array}\right),\;v={\bf 
   (}(v)^\dagger_+{\bf ,} \; (v)^\dagger_-{\bf )}\;.
\end{equation}
All four cameras had scales of the order of 0.4 arc 
sec/pixel, and our field sizes were around 3 arc min.  
The Weyl spinors are given in terms of helicity 
eigen\-states $\chi_\lambda(p)$ with $\lambda=\pm1$ by
\begin{eqnarray}
   u(p, \lambda)_\pm & = & 
   (E\pm\lambda|{\bf p}|)^{1/2}\chi_\lambda(p)\;, 
   \nonumber \\ & & \\v(p,\lambda)_\pm & = & 
   \pm\lambda(E\mp\lambda|
   {\bf p}|)^{1/2}\chi_{-\lambda}(p). \nonumber
\end{eqnarray}

\subsubsubsection{Level four heads capitalize only 
the first letter of the first word, indent one-em 
space, and end with a colon}  
These spinors are expressed in terms 
of two-component Weyl spinors as 
\begin{equation}
   u=\left(\begin{array}{c}(u)_-\\(u)
   _+\end{array}\right),\;v={\bf (}(v)^\dagger_+{\bf ,} 
   \; (v)^\dagger_-{\bf )}\;.
\end{equation}
The Weyl spinors are given in terms of helicity 
eigen\-states $\chi_\lambda(p)$ with $\lambda=\pm1$ 
by
\begin{eqnarray}u(p, \lambda)_\pm & = & 
   (E\pm\lambda|{\bf p}|)^{1/2}\chi_\lambda(p)\;, 
   \nonumber \\ & & \\v(p,\lambda)_\pm & = & 
   \pm\lambda(E\mp\lambda|{\bf 
   p}|)^{1/2}\chi_{-\lambda}(p). \nonumber
\end{eqnarray}
The CCD images are unfortunately not always suitable 
for very poor clusters or for clusters with large cores.

%% ------------------------------------------------------ %%
   %
   % In the next sections we have a reference to one of 
   % the tables (tables appear at the end of the LaTeX 
   % document).  There is also some additional math-related 
   % markup.  In the second paragraph, note the use of 
   % in-text math material between the dollar signs ($$) 
   % including a couple of the miscellaneous symbol 
   % commands defined in AGU's style file package.
   % 
   % Note that keyboard math commands (such as greater 
   % than or less than) must appear within a math or 
   % equation environment or they will not print properly 
   % (i.e., $<$ or $>$).  All other math symbols (such as 
   % similar) must be created with appropriate LaTeX symbol 
   % commands (i.e., $\sim$ instead of ~) or they will not 
   % print properly.  See the aguguide.tex document for a 
   % complete list of keyboard and LaTeX math symbols.
   %
%% ------------------------------------------------------ %%

\section{Floating Material}
Consider a task that computes profile parameters for 
a modified Lorentzian of the following form:
\begin{equation}
   I = \frac{1}{1 + d_{1}^{P (1 + d_{2} )}}, 
\end{equation}
where
\begin{mathletters}
\begin{eqnarray}
   d_{1} = \frac{3}{4} \sqrt{ \left( \begin{array}{c} 
   \frac{x_{1}}{R_{\rm maj}} 
   \end{array} \right) ^{2} + \left( 
   \begin{array}{c} \frac{y_{1}}{R_{\rm min}} 
   \end{array} \right) ^{2} } \\
   d_{2} = \case{3}{4} \sqrt{ \left( 
   \begin{array}{c} \frac{x_{1}}{P R_{\rm maj}}
   \end{array} \right) ^{2} + \left( 
   \begin{array}{c} \case{y_{1}}{P R_{\rm min}} 
   \end{array} \right) ^{2}.}
\end{eqnarray}
\end{mathletters}
In these expressions, $x_{0}$, $y_{0}$ is the star 
center, and $\Theta$ is the angle with the $x$ axis.
Results of this task are shown in 
   %
%% ------------------------------------------------------ %%
%
%%  CALLOUTS
%
%% ------------------------------------------------------ %%
   %
\callout{Table~\ref{tbl-1}}
and
\callout{Table~2}.
   %
   % The first time a table, figure, or plate is 
   % mentioned in a manuscript, AGU asks that it 
   % be called out (i.e., identified in the right 
   % margin of the camera-ready copy).
   %
   % Enclose the first occurrence of a table, figure,
   % or plate within a \callout{} command.  Text within
   % this command will appear in your manuscript text 
   % and will also appear in a box in the right margin 
   % of your manuscript.  Make sure all callouts appear
   % in numerical sequence (i.e., Table 1, Table 2, etc.)
   %
   % You may use \ref{} commands within callout commands,
   % as shown above.
   %
It is not clear how these sorts of analyses may affect 
determination of $M_{\sun}$ and $M_{\earth}$, but the 
assumption is that the alternate results should be less 
than 90\deg\ out of phase with previous values.  We have 
no observations of \ion{Ca}{2}.


%% ------------------------------------------------------ %%
%%
%%  APPENDICES
%%
%%

\appendix
\section{Appendix A: Your Title}

   % If you have several appendix sections, lettering in
   % the appendix header should match the above example.
   % If there is only one appendix, lettering should not
   % appear in the header: \section{Appendix: Your Title}.

   % Important:  The \appendix command causes all 
   % following equations to use a lettered number
   % (for instance, A1).  This is to distinguish 
   % appendix equations from equations that appear 
   % in the main body of the manuscript.

   % Appendix tables must be manually labeled.
   % If your manuscript contains tables that 
   % accompany an appendix section, you must use 
   % \tablenum{} commands and label the table(s) 
   % accordingly.  (If there is one appendix section 
   % and one table, the table number will be 
   % \tablenum{A1}; if there are two appendix 
   % sections, tables appearing in the second 
   % section will begin with \tablenum{B1}, etc.)

Consider a task that computes profile parameters for 
a modified Lorentzian of the form
\begin{equation}
   I = \frac{1}{1 + d_{1}^{P (1 + d_{2} )}}, 
\end{equation}
where
\begin{mathletters}
\begin{equation}
   d_{1} = \frac{3}{4} \sqrt{ \left( \begin{array}{c} 
           \frac{x_{1}}{R_{maj}} 
   \end{array} \right) ^{2} + \left( \begin{array}{c} 
           \frac{y_{1}}{R_{min}} 
   \end{array} \right) ^{2} } 
\end{equation}
\begin{equation}
   d_{2} = \case{3}{4} \sqrt{ \left( \begin{array}{c} 
           \frac{x_{1}}{P R_{maj}}
   \end{array} \right) ^{2} + \left( \begin{array}{c} 
           \case{y_{1}}{P R_{min}} 
   \end{array} \right) ^{2} }  
\end{equation}
which leaves us with the conclusion that
\begin{equation}
   x_{1} = (x - x_{0}) \cos \Theta 
   + (y - y_{0}) \sin \Theta 
\end{equation}
\begin{equation}
   y_{1} = -(x - x_{0}) \sin \Theta 
   + (y - y_{0}) \cos \Theta. 
\end{equation}
\end{mathletters}

\section{Appendix B: One Last Equation}
For completeness, here is one last equation
\setcounter{equation}{0}  
\begin{equation}
e = mc^2.
\end{equation}
Notice how the equation numbering and lettering 
is reset when a new appendix section is added.


%% ------------------------------------------------------ %%
%
%%  NOTATION TABLES (if any)
%
%% ------------------------------------------------------ %%
   %
   % If your manuscript contains notation tables, use a 
   % Level 1 head for the word "Notation" and use a 
   % tabular environment without any rules for the 
   % notation information, as shown below.
   %
   % If you numbered your sections, place a 
   % \setcounter{secnumdepth}{0} in front of 
   % the Notation section head, so it will 
   % appear flush left and without numbers.
   %
   % Set the symbol columns flush right and the 
   % definition column flush left.  You will probably
   % need to adjust the {rp{17.5pc} spacing to create
   % appropriate widths for your notation columns.

\setcounter{secnumdepth}{0}
\section{Notation}
\begin{tabular}{rp{17.5pc}}
\hspace{-2em} b                                        &
type notation text describing the letter ``b.''        \\
\hspace{-2em} a$^{\prime}$                             &
type notation text describing ``a$^{\prime}$.''        \\
\hspace{-2em} f                                        &
type notation text describing the letter ``f.''        \\
\hspace{-2em} O$_2$                                    &
type notation text describing ``O$_2$.''               \\
\hspace{-2em} {\bf K}                                  &
type notation text describing the boldface ``{\bf K}.''\\
\hspace{-2em} g                                        &
type notation text describing the letter ``g.''        \\
\end{tabular}


%% ------------------------------------------------------ %%
%
%%  ACKNOWLEDGMENTS (optional)
%
%% ------------------------------------------------------ %%

\acknowledgments
 We are grateful to V. Barger, \linebreak
T.~Han, and R. J. N. Phillips for doing 
the math in the formalism section.

The Editor would like to thank the reviewer of 
this manu\-script.

   % Type your acknowledgment text (if any) 
   % immediately after the \acknowledgments 
   % command.  Acknowledgments may be only 
   % one paragraph in length.  
   %
   % Be sure to spell "\acknowledgments" 
   % exactly as it appears above; if you spell 
   % it differently then LaTeX will not recognize 
   % the command and it will not work.  There is 
   % no "end acknowledgments" command.
   %
   % If your manuscript contains only one 
   % acknowledgment, use an \acknowledgment 
   % command to produce a singular 
   % "Acknowledgment" subhead.
   %
   % If your manuscript contains Editor's 
   % acknowledgments, set them in a new 
   % paragraph one line below the regular 
   % acknowledgments, as shown.

%% ------------------------------------------------------ %%

% \newpage

   % If desired, you may place a \newpage command after 
   % the acknowledgments section; this will force the 
   % references section to begin on a new page.


%% ------------------------------------------------------ %%
%
%%  REFERENCES
%
%% ------------------------------------------------------ %%

   % In this document we use \markcite commands to refer
   % to citations, so we must enclose references in a 
   % "references" environment.  

        % Include all references between the 
   % \begin{references} and \end{references} 
      % commands.  Each reference must be preceded 
        % by a \reference command.
        % 
       % Extra spacing before each reference is not 
   % necessary; it will not appear in the final 
   % version and is used here only to make the 
        % examples easier to read.  
        %
       % The type of reference being shown appears 
    % in parentheses (after the percent sign) 
      % for each of the following examples; that 
        % information does not need to appear in 
        % your manuscripts or your input files.
        %
    % References should follow the sample format, 
  % with journal titles and issue numbers in 
        % italic type. 
        %
     % Markup commands have been created for some 
        % of the journals referenced most often (such
        % as \jgr or \grl).  Authors may use these 
        % commands as shorthand rather than type out 
        % the whole journal name.  See the aguguide.tex
        % for a full list of short commands.  Note that
        % a comma is automatically included after each
        % journal listing.

\begin{references}
\reference % (Journal article)
Baker, K. D., D. J. Baker, J. C. Ulwick, and A. T. 
Stair Jr., Infrared enhancements associated with a 
bright auroral breakup, \jgr {\it 82,} 
3518-3528, 1977.

\reference % (Edited book)
Barbour, M. G., and J. Major (Eds.), {\it Terrestrial 
Vegetation of California,} 1002 pp., John Wiley, New 
York, 1977.

\reference % (Authored book)
Bell, R. J. {\it Introductory Fourier Transform 
Spectroscopy,} 329 pp., Academic, San Diego, Calif., 
1972.

\reference % (Article in book)
Brosche, P., and J. Sunderman, Effects of oceanic 
tides on the rotation of the Earth, in {\it Scientific 
Applications of Lunar Laser Ranging,} edited by J. D. 
Mulholland, pp. 133-141, D. Reidel, Norwell, Mass., 1977.

\reference % (Laboratory or technical report)
Degges, T. C., and H. J. P. Smith, A high altitude 
infrared radiance model, {\it Tech. Rep. AFGL-TR-77-02721, 
AD-A059242,} 25 pp., Air Force Geophys. Lab., Bedford, 
Mass., 1977.

\reference % (Article in a translation journal)
Golitsyn, G., and I. Mokhov, Stability and extremal 
properties of climate models, {\it Izv. Russ. Acad. Sci. 
USSR Atmos. Oceanic Phys.,} Engl. Transl., {\it 31,} 
781-787, 1995.

\reference % (Map)
Higgins, M. W., Geologic map of the Brevard fault 
zone near Atlanta, Georgia, scale 1:48,000, {\it U.S. 
Geol. Surv. Geol. Invest. Map, I-511,} 1968.

\reference % (Thesis)
McMullen, R. J., The effect of geothermal gradients on 
nonlinear creep deformations in the lithosphere, M.A. 
thesis, 256 pp., State Univ. of N.Y. at Buffalo, May 1978.

\reference % (Meeting paper)
Namias, J., Suggestions for research leading to long range 
precipitation forecasting for the tropics, paper presented 
at International Tropical Meteorology Meeting, Am. Meteorol.
Soc., Nairobi, Jan. 31 to Feb. 7, 1974.

\reference % (Abstract)
Poppe, B., and R. Zwickl, Internet education by Space 
Environment Laboratory (abstract), \eos {\it 76}(17), 
Spring Meet. Suppl., S210, 1995.

\reference % (Proceedings paper)
Stair, A. T., Jr., Cryogenic spectrometry for the measure 
of airglow and aurora, {\it Proc. Soc. Photo. Opt. Instrum. 
Eng., 91,} 71-75, 1976.

\reference % (Nonroman alphabet article)
Yamazaki, Y., Resistivity change at Aburatsubo caused by 
the earthquake of magnitude 7.0 on January 17, 1978 (in 
Japanese), {\it J. Seismol. Soc. Jpn., 31,} 230-233, 1978.
\end{references}

\end{article}
   % You must type an \end{article} 
   % command after the references.

\newpage

   % An alternate way of handling references is to use 
   % "\cite" and "\bibitem" commands to call out citations,
   % and then use LaTeX's "thebibliography" environment 
   % for the reference list.  Please note that the 
   % \begin{thebibliography}{} command is followed by a 
   % null argument {} (see the example below).
   %
   % Each reference has a \bibitem command to define 
   % the citation format and the symbolic tag, as well 
   % as a \reference command which sets up formatting 
   % parameters for the reference list itself.  Authors 
   % choose what markup appears in the text, but it must 
   % match the tags used with the corresponding \bibitem 
   % command.  (For example, [\cite{can95}; \cite{kur94};
   % \cite{hag86}] or [\cite{aur92}].)
   %
   % A drawback to using this system is that 
   % references will be typeset exactly as they 
   % appear in the \bibitem command.  Thus the 
   % \cite-\bibitem system is unable to produce 
   % format such as {\it Kurtz and King} [1994].  
   % If your manuscript ever uses this format you 
   % should probably use the \markcite commands.
   %
   % The following examples are commented out so 
   % the file will LaTeX properly.


%%%% Examples of \cite Markup Commands in Text:
   %
   %    These were studied by \cite{kur94} and 
   %    \cite{can95}.  % Note the {\it a, b} format below.
   %    However, they were thought to be insignificant 
   %    \cite[1993]{aur92}.  
   %    The data made it necessary to reconsider earlier 
   %    findings \cite{kur94, hag93}.


%%%% If you cite two references from two different years,
   % you must type only one \cite in the input file and
   % include the second year in square brackets, as shown
   % above, in order to output correct AGU reference style.

%%%% Examples of \cite Markup Commands in thebibliography:
   % 
   % \begin{thebibliography}{}
   % \bibitem[{\it Auri\`ere,} 1992]{aur92} 
   % \reference 
   % Auri\`ere, P., Title of article one,
   % \apj {\it , 109,} 301-305, 1992.
   % 
   % \bibitem[{\it Auri\`ere,} 1993]{aur93} 
   % \reference 
   % Auri\`ere, P., Title of article two,
   % \apj {\it 110,} 173-177, 1993.
   % 
   %
 %%%%  If you cite two references in the same year at
    %  the same time, you must use the {\it a, b} format
    %  shown below and mark up the \cite command in the 
 %%%%  text as shown in the {can95} text sample, above.
   %
   %
   % \bibitem[{\it Cannon et al.,} 1995{\it a, b}]{can95}
   % \reference 
   % Cannon, W. M., K. Lille, T. Hanley, B. James,
   % M. John, and J. J. Stanco.
   % Title of article, \apj {\it 137,} 52, 1995a.
   % 
   % \bibitem[{\it Cannon et al.,} 1995{\it b}]{can95b}
   % \reference 
   % Cannon, W. M., J. McLendon, M. T. Jeffries, and 
   % D. L. S. Walker
   % Title of article, \grl {\it 22,} 18, 1995b.
   % 
   % \bibitem[{\it Kurtz and King,} 1994]{kur94} 
   % \reference 
   % Kurtz, S., and E. R. King, Title of article, 
   % \jgr {\it 21,287,} 99, 1994.
   % 
   % \bibitem[{\it Hagiwara and Zeppenfeld,} 1993]{hag93} 
   % \reference 
   % Hagiwara, K., and D. Zeppenfeld, Title of article, 
   % \rs {\it 274,} 31, 1993. 
   % \end{thebibliography}
   % \end{article}
          % You must type an \end{article} command 
          % after the references.

%% ------------------------------------------------------ %%
%
%%  NO REFERENCE SECTION (forcesluginfo)
%
%% ------------------------------------------------------ %%

   % If your manuscript contains no reference section, 
   % you must place a \forcesluginfo command in the 
   % manuscript after the last paragraph of text (or 
   % after the acknowledgments, if any).  Otherwise,
   % your manuscript information will not print.

% \forcesluginfo

%% ------------------------------------------------------ %%
%
%%  FIGURE CAPTIONS
%
%% ------------------------------------------------------ %%

   % Set figure captions between the curly brackets in the 
   % \caption command.  Each figure caption must have a 
   % \begin{figure} before and an \end{figure} after, 
   % as shown.  Several captions may be printed per page, 
   % as long as there is sufficient room to cut between 
   % the captions.  
   %
   % It may be necessary to use \clearpage commands 
   % between a long series of captions, since a large 
   % number of captions can cause LaTeX's memory buffer 
   % to overload and crash.  
   %
   % AGU asks authors to submit two sets of captions; 
   % one set for single column and one set for double 
   % column figures.  You must input the first set of 
   % captions, but AGU's LaTeX style files will 
   % automatically generate the second set of captions.  
   % Both sets of captions will print at the correct 
   % single- and double-column widths for your journal 
   % (these widths change depending on the document style 
   % you choose).
   %
   % The \figurewidth commands do not affect the second
   % set of figure captions.  The second set of captions
   % will always mirror the figure numbers of the first
   % set of captions.
   %
   % The wide figure captions will likely cause
   % "overfull hbox" error messages to appear when 
   % you LaTeX your file: please ignore these.
   %
   % Be aware that figure width and figure numbering 
   % commands may affect following plate captions, 
   % and vice versa.
   %
   % Caption lines may be broken by using a \protect 
   % command in front of a \linebreak command:
   % \protect\linebreak (this will maintain justification).

\begin{figure}
        \caption{We use the \LaTeX\ {\tt figure} environment 
        to set figure captions.  Figure captions consist of 
        a paragraph containing several sentences or phrases.}
\end{figure}

       \begin{figure}
\figurenum{17c}
   % The \figurenum command allows you to explicitly 
      % number your figure captions.  Letters or symbols 
     % may also be included.  This command *does* affect 
    % LaTeX's figure counter (so the next figure number 
    % would be 3 instead of 2, if there was no explicit
     % \figurenum or \setcounter command).
   % 
\figurewidth{13pc}
    % The \figurewidth command, when used inside 
   % a figure environment, sets the width of the 
  % first set of that particular caption to the 
        % specified width (shown is 13 picas).  
  % 
      \caption{AGU asks authors to submit two sets 
        of captions.  Authors may change the width of 
        the first set of captions using a ``figurewidth'' 
        command, but the second set of captions will 
        always print at the correct double-column width 
        for your journal.  The first caption for this 
        figure will print at a width of 13 picas.}
      \end{figure}


\setcounter{figure}{0}  
        % (set this to 0 if the next number is 1, etc.)
     % To reset the figure counter, place this command 
      % in front of your second set of figure captions.

\figurewidth{25pc}
     % The \figurewidth command, when placed in front of a 
  % series of figure captions, sets all following captions 
       % to the specified width (shown is 25 picas).

   \begin{figure}
  \caption{We use the \LaTeX\ {\tt figure} 
       environment to set another figure caption.  
    This caption was preceded by commands that
        reset the figure number and the figure width.
        The figure number is reset to ``one'' and the 
        first figure caption width is set to 25 picas.}
    \end{figure}

%% ------------------------------------------------------ %%
%
%%  PLATE CAPTIONS
%
%% ------------------------------------------------------ %%

   % Plate captions are similar to figure captions, 
   % but they use \platenum{}, \platewidth, and 
   % \setcounter{plate}{} commands.  Again, AGU 
   % asks authors to submit two sets of captions; 
   % one set for single column and one set for 
   % double column plate widths.
   %
   % It may be necessary to use \clearpage commands 
   % between a long series of captions, since a large 
   % number of captions can cause LaTeX's memory buffer 
   % to overload and crash.  Be aware that plate commands 
   % may affect following figure captions, and vice versa.

 \begin{plate}
\platenum{7c}
      % The \platenum command allows you to explicitly number 
        % your plate captions.  Letters or symbols may also be 
 % included.  This command *does* affect LaTeX's plate 
  % counter (so the next plate number would be 3, not 2).
 % 
\platewidth{41pc}
     % The \platewidth{} command, when used inside a plate 
  % environment, sets the width of that particular caption 
       % to the specified width (shown is 41 picas).
   % 
      \caption{Use a \LaTeX\ {\tt plate} environment
   to set plate captions.  This plate caption 
     contains commands that reset the plate number 
         to ``7c'' and the first plate caption width 
         to 41 picas.  Figure captions accompany black 
         and white line art or gray scale artwork, 
         while plate captions accompany color artwork.}
   \end{plate}


%% ------------------------------------------------------ %%
%
%%  TABLES
%
%% ------------------------------------------------------ %%

%%%% Table Page Breaks
   %
   % Tables should be submitted one or two to a page 
   % following the main body of the text.  (Planotables 
   % automatically begin on a new page for formatting 
   % reasons.)  Use \clearpage commands to force page 
   % breaks as needed.  Include a \clearpage command 
   % after the last table so it is forced onto its own 
   % page as well.

%%%% Table Rules
   %
   % AGU table style allows no vertical rules.  The 
   % required horizontal rules are created automatically
   % by the \startdata command.  If you must include 
   % other rules, use a \tableline command (not an 
   % \hline command).

%%%% Table Footnotes
   % 
   % Some column headings require special annotation, 
   % i.e., table footnotes.  These are marked and tagged 
   % with \tablenotemark{tag}.  The \tablenotemark commands
   % may be placed on individual data entries as well, 
   % but try to keep them to a minimum.  Accompanying text 
   % appears in \tablenotetext{tag}{TEXT} commands; the 
   % \tablenotetext "tag" must match the "tag" in the 
   % appropriate \tablenotemark command.
   % 
   % Place short table notes in \tablenotetext{\null}{TEXT} 
   % commands, with a \null argument as the tag.  Longer 
   % notes may be placed in a \tablecomment{TEXT} command.  
   % Lists of table references should use this format:
   % \tablecomment{References:  Names of your references.}  
   % There are examples of these in the tables below.

%%%% Column Headers
   % 
   % The \colhead{} command automatically creates centered 
   % column headers, in accordance with AGU style.  Use this
   % command in place of the usual \multicolumn{} command.  
   % If you prefer to use multicolumn{}, or if you have a 
   % more complicated table, AGU's style files will still 
   % accept that coding.  It is not recommended that both 
   % the \colhead{} and \multicolumn{} commands be used in 
   % the same table.  Examples of both styles are included.

%%%% Table Captions 
   %
   % Captions should contain only the caption text.  
   % The "Table #." identification is generated 
   % automatically.  If you need to change the table 
   % number, see the section on table numbers, below.
   %   
   % Use \tablecaption{} commands with planotables and
   % \caption{} commands with tabular tables.

%%%% Tabular Tables
   %
   % It is possible to use a "tabular" environment to 
   % create tables.  The tabular environment should be 
   % embedded within a table environment (which is needed 
   % for automatic numbering and caption generation).  
   % See the last table for an example.

%%%% Planotables
   %
   % Planotables have several capabilities (above and 
   % beyond LaTeX's tabular environment) that facilitate
   % the formatting of tables.  For instance, it is 
   % possible to break long planotables across pages, 
   % and while tabular tables print at an automatic width,
   % authors may choose a specific planotable width.
   % See the aguguide.tex and samples below for more 
   % information.



%%%% Numbering Tables
   % 
   % If you must create a particular table number, or if 
   % you need a letter before or after a table number,
   % use a \tablenum{} command, with the desired text
   % between the curly brackets.  Several examples are
   % shown below.  A \tablenum command will affect any
   % following table numbers.

%%%% Setting Planotable Widths
   %
   % Use \tablewidth{} commands to set the width of 
   % your planotables.  See the aguguide.tex or the
   % Table section in your General Instructions to
   % choose an appropriate size for your table.

%% ------------------------------------------------------ %%

\begin{planotable}{crrrrrrrrrrr}
\tablewidth{41pc}
\tablecaption{Planotable with Examples of Footnotes}
\tablenum{1}
\tablehead{
      \colhead{Event}           &
     \colhead{Height, km}      &
     \colhead{$d_{x}$}         &
     \colhead{$d_{y}$}         &
     \colhead{$n$}             &
     \colhead{$\chi^2$}        &
     \colhead{$R_{\rm maj}$}   &
     \colhead{$R_{\rm min}$}   &
     \colhead{$P$\tablenotemark{a}} &
        \colhead{$P R_{\rm maj}$} &
     \colhead{$P R_{\rm min}$} &
     \colhead{$\Theta$\tablenotemark{b}}
}
   % Text for table footnotes should directly precede the 
   % "startdata" command.  If they are placed between the 
   % table data and the \end{planotable} command then the 
   % last cell of information will not indent properly.  
   % Note that it is acceptable to use \ref commands in 
   % \tablenotetext commands.
\tablenotetext{a}{Sample footnote for 
                      Table~\ref{tbl-1}.}
\tablenotetext{b}{Another sample footnote for 
                      Table~\ref{tbl-1}.}
\startdata
\label{tbl-1}
1 &33472.5 &-0.1 &0.4  &53 &27.4 &2.065   &
1.940 &3.900 &68.3 &116.2 &-27.639        \nl
2 &27802.4 &-0.3 &-0.2 &60 &3.7  &1.628   &
1.510 &2.156 &6.8  &7.5 &-26.764          \nl
3 &29210.6 &0.9  &0.3  &60 &3.4  &1.622   &
1.551 &2.159 &6.7  &7.3 &-40.272          \nl
4 &32733.8 &-1.2 &-0.5 &41 &54.8 &2.282   &
2.156 &4.313 &117.4 &78.2 &-35.847        \nl
5 & 9607.4 &-0.4 &-0.4 &60 &1.4  &1.669   &
1.574 &2.343 &8.0  &8.9 &-33.417          \nl
6 &31638.6 &1.6  &0.1  &39 &315.2 & 3.433 &
3.075 &7.488 &92.1 &25.3 &-12.052
\end{planotable}


\begin{planotable}{ccccc}
\tablewidth{20pc}
\tablecaption{Planotable With Multicolumn Commands}
\tablenum{33a}
\tablehead{\multicolumn{1}{c}{}&
           \multicolumn{2}{c}{Length}&
           \multicolumn{2}{c}{Distance}\\[.3ex]
        \cline{2-3}
        \cline{4-5}\\[-1.6ex]
           \multicolumn{1}{c}{} &
           \multicolumn{1}{c}{0 mK} &
           \multicolumn{1}{c}{3 mK} &
           \multicolumn{1}{c}{0 mK} &
           \multicolumn{1}{c}{3 mK}
}
\tablenotetext{\null}{Occasionally, authors may wish to 
            append a short paragraph of explanatory notes 
            which pertain to an entire table but are 
            different than the caption.  Such notes may 
            be placed in a ``tablenotetext'' environment 
            with a ``null'' argument as demonstrated in 
            this example.}
\startdata
Height off & 20 & 21 & 22 & 23\nl
Seabed, m  & 10 & 11 & 12 & 13
\end{planotable}


\begin{planotable}{lrrrrcrrrrr}
\tablewidth{35pc}
\tablecaption{Literature Data for Program Stars}
\tablenum{7}
\tablehead{
 \colhead{Star}   & 
     \colhead{V}      &
      \colhead{b$-$y}  & 
     \colhead{m$_1$}  &
      \colhead{c$_1$}  & 
     \colhead{Source} &
      \colhead{T$_{\rm eff}$} & 
      \colhead{log g}  &
      \colhead{v$_{\rm turb}$} & 
     \colhead{[Fe/H]} &
      \colhead{Source}
}
\tablecomments{Sources:  
1, {\it Barbuy et al.} [1985];
2, {\it Bond} [1980];
3, {\it Carbon et al.} [1987];
4, {\it      Hobbs and Duncan} [1987];
5, {\it Gilroy et al.} [1988];
6, {\it Gratton and Ortolani} [1986];
7, {\it Gratton and Sneden} [1987, 1988, 1991];
8, {\it Kraft et al.} [1982];
9, {\it Laird} [1990];
10, {\it Leep and Wallerstein} [1981];
11, {\it Luck and Bond} [1981, 1985];
12, {\it Magain} [1987, 1989];
13, {\it Peterson} [1981];
14, {\it Peterson et al.} [1990];
15, {\it Royce et al} [1988];
16, {\it Schuster and Nissen} [1988a, b];
17, {\it Spite et al.} [1984];
18, {\it Spite and Spite} [1986];
19, {\it Hobbs and Thorburn} [1991].  Only one 
    paragraph of material is permitted at the end
    of a table (excluding superscripted footnotes),
    so if both references and notes exist, they should
    be run in together.}
\tablenotetext{a}{This is another example
                      of a footnote.}
\tablenotetext{b}{Star LP 608--62 is also known
                      as BD+1\deg 2341p.}
\startdata
HD 97 & 9.7& 0.51& 0.15& 0.35& 8 & \nodata & \nodata &
\nodata & $-1.50$ & 2 \nl
& & & & & & 5015 & \nodata & \nodata & $-1.50$ & 10 \nl
& & & & & & 5000 & 2.50 & 2.4 & $-1.99$ & 5 \nl
& & & & & & 5120 & 3.00 & 2.0 & $-1.69$ & 7 \nl
& & & & & & 4980 & \nodata & \nodata & $-2.05$ & 10 \nl
HD 4306 & 9.0& 0.52& 0.05& 0.35& 19, 2& \nodata & \nodata &
\nodata & $-2.70$ & 2 \nl
& & & & & & 5000 & 1.50 & 1.8 & $-2.65$ & 14 \nl
& & & & & & 4950 & 2.10 & 2.0 & $-2.92$ & 8 \nl
& & & & & & 5000 & 2.25 & 2.0 & $-2.83$ & 18 \nl
& & & & & & \nodata & \nodata & \nodata & $-2.80$ & 1 \nl
HD 84937 & 8.3& 0.30& 0.06& 0.35& 4, 11& 6200 & \nodata &
\nodata & $-2.10$ & 4 \nl
& & & & & & 6216 & \nodata & \nodata & $-2.42$ & 11 \nl
& & & & & & 6240 & \nodata & \nodata & $-2.13$ & 3 \nl
& & & & & & \nodata & \nodata & \nodata & $-2.14$ & 1 \nl
& & & & & & 6200 & 3.60 & 1.5 & $-2.43$ & 16 \nl
& & & & & & 6250 & 4.00 & \nodata & $-2.10$ & 3 \nl
%
\cutinhead{This Is a Center Head} \nl
%
HD 87140 & 9.0& 0.48& 0.12& 0.28& 7 & 5000 & 4.50 &
1.0 & $-1.41$ & 7 \nl
& & & & & & \nodata & \nodata & \nodata & $-1.56$ & 1 \nl 
%
\tablebreak & & & & & & 4500 & 1.10 & 2.8 & $-2.77$ & 5 \nl 
%
HD 94028 & 8.2& 0.34& 0.08& 0.25& 12, 15 & 5795 & 4.00 &
\nodata & $-1.70$ & 3 \nl
& & & & & & 4500 & 0.80 & 3.2 & $-2.65$ & 14 \nl
& & & & & & 4600 & \nodata & \nodata & $-2.75$ & 10 \nl
& & & & & & 5860 & \nodata & \nodata & $-1.70$ & 4 \nl
& & & & & & 5910 & 3.80 & \nodata & $-1.76$ & 15 \nl
& & & & & & 5900 & \nodata & \nodata & $-1.57$ & 3 \nl
& & & & & & \nodata & \nodata & \nodata & $-1.32$ & 1 \nl
HD 97916 & 9.2& 0.29& 0.10& 0.41& 13, 14 & 6125 & 4.00 &
\nodata & $-1.10$ & 3 \nl
& & & & & & 5950 & \nodata & \nodata & $-1.50$ & 17 \nl
& & & & & & 6204 & \nodata & \nodata & $-1.36$ & 11 \nl
%
\cutinhead{At Least Two Center Heads 
Are Required if Any Are Used} \nl
%
+26\deg2606& 9.7&0.34&0.05&0.28&5, 6& 5980 & \nodata & 
\nodata & $<-2.20$ & 19 \nl
& & & & & & 5950 & \nodata & \nodata & $-2.89$ & 9 \nl
+26\deg3578& 9.4&0.31&0.05&0.37&10& 5830 & \nodata &
\nodata & $-2.60$ & 4 \nl
& & & & & & 5800 & \nodata & \nodata & $-2.62$ & 17 \nl
& & & & & & 6177 & \nodata & \nodata & $-2.51$ & 11 \nl
& & & & & & 6000 & 3.25 & \nodata & $-2.20$ & 3 \nl
& & & & & & 6140 & 3.50 & \nodata & $-2.57$ & 15 \nl
& & & & & & 4260 & \nodata & \nodata & $-1.55$ & 10 \nl
+37\deg1458& 8.9&0.44&0.07&0.22&9, 16& 5296 & 
\nodata & \nodata & $-2.39$ & 11 \nl
& & & & & & 5420 & \nodata & \nodata & $-2.43$ & 3 \nl
& & & & & & 5000 & 1.10 & 2.2 & $-2.71$ & 14 \nl
& & & & & & 5000 & 2.20 & 1.8 & $-2.46$ & 5 \nl
G5--36\tablenotemark{a} & 10.8& 0.40& 0.07& 
0.28& 3, 17 & \nodata & \nodata & \nodata & $-1.19$ & 1 \nl
& & & & & & 4980 & \nodata & \nodata & $-2.55$ & 10 \nl
& & & & & & \nodata & \nodata & \nodata & $-2.03$ & 1 \nl
& & & & & & 6020 & \nodata & \nodata & $-1.56$ & 3 \nl
LP 608--62\tablenotemark{b} & 10.5& 0.30& 0.07& 0.35 &
1, 18 & 6250 & \nodata & \nodata & $-2.70$ & 4
\end{planotable}

\begin{planotable}{llrr}
\tablewidth{20pc}
\tablecaption{Example of an Appendix Table}
\tablenum{A1}
\tablehead{  \multicolumn{2}{c}{} &
             \multicolumn{2}{c}{\it dl/dt, \rm mm/yr}\\[.3ex]
\cline{3-4}\\[-1.6ex]
             \multicolumn{1}{c}{From} &
             \multicolumn{1}{c}{To} &
             \multicolumn{1}{c}{Observed} &
             \multicolumn{1}{c}{Model}
}
\tablecomments{Occasionally, authors may wish to append 
    longer paragraphs of explanatory notes which pertain 
    to an entire table but are different from the caption.  
    These notes may be placed in a ``tablecomments'' 
    environment as demonstrated in this example.  Note 
    that this table uses the ``tablenum'' command since 
    it needs a letter ``A'' in the table caption.}
\startdata
Alamillo & Palvadero & $0.6 + 0.8$ & $1.4$\nl
Campana & Canas &$0.4 + 1.1$ & $-0.7$\nl
& Chupadera & $ -0.5 + 1.0$ & $0.2$
\end{planotable}

\vspace{6pc}

\begin{table}
\caption{Example of a Table in a Tabular \protect\\ 
Environment} 
\tablenotetext{}{This table is an example 
of \LaTeX's \verb"table" and \verb"tabular" 
environments.  Blank columns have been added
to force the table width closer to 20 picas.}
\vspace{5pt}
\begin{tabular}{llllrrr} 
\tableline 
& & & \\[-5pt]
      \multicolumn{4}{c}{} & 
 \multicolumn{3}{c}{ \it dl/dt, \rm mm/yr}\\[4pt]
\cline{4-7}\\[-7pt]
     \multicolumn{1}{c}{From} & 
     \multicolumn{1}{c}{} & 
 \multicolumn{1}{c}{To} & 
       \multicolumn{1}{c}{} & 
 \multicolumn{1}{c}{Observed} & 
 \multicolumn{1}{c}{} & 
 \multicolumn{1}{c}{Model} \\[4pt]
\tableline
& & & & & & \\[-6pt]
Alamillo & & Palvadero & & $0.6 + 0.8$ & & $1.4$ \\
Campana & & Canas & & $0.4 + 1.1$ &&  $-0.7$\\
& & Chupadera &&  $ -0.5 + 1.0$ & & $0.2$\\[4pt]
\tableline
& & & & & & \\[-6pt]
\end{tabular}
\end{table}

   % This is the last table for this paper so we should 
   % follow it with a \clearpage command.  The \clearpage 
   % command forces all the floating tables out of the
   % memory buffers.

\clearpage

%% ------------------------------------------------------ %%
%
%%  Abstract Supplement (for JGR-Space Physics only)
%
%% ------------------------------------------------------ %%

% \printabstract

      % The \printabstract command works only with the
        % jgrga.sty; it will cause a LaTeX error message
        % in all other substyles.
       % 
      % JGR-Space Physics requires authors to submit 
 % an abstract supplement with each camera-ready 
        % manuscript.
   %
       % To make LaTeX generate an abstract supplement, 
       % type a \printabstract command in front of the 
        % \end{document} command.  LaTeX will then copy 
        % any text within the \begin{abstract} and 
     % \end{abstract} environment, format it in proper
       % Abstract Supplement style, and print it at the 
       % end of your LaTeX document.


\end{document}

     % All LaTeX input files must conclude with an 
  % \end{document} command.  LaTeX cannot process 
        % a file that does not contain this command.
