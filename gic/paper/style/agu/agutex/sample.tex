%%%%%%%%%%%%%%%%%%%%%%%%%%%%%%%%%%%%%%%%%%%%%%%%%%%%%%%%%%%%%%%%%%%%%%%%%%%%
% Template.tex: this template is for journals formatted with LaTeX2e,
% Modified 15 May 2003
%
% This template is set up logically, with commands and instructions
% given in the order necessary to produce a final output that will
% satisfy AGU requirements.
%
% PLEASE DO NOT USE YOUR OWN MACROS
%
% All questions should be e-mailed to author.help@agu.org
%%%%%%%%%%%%%%%%%%%%%%%%%%%%%%%%%%%%%%%%%%%%%%%%%%%%%%%%%%%%%%%%%%%%%%%%%%%%

%% ------------------------------------------------------------------------ %%
%Revision History
%
%BibTeX instructions added to \bibliography section (15 May 2003)
%% ------------------------------------------------------------------------ %%


%% ------------------------------------------------------------------------ %%
%
%  First set the documentclass and choose the journal style.
%
%% ------------------------------------------------------------------------ %%

% EXAMPLES
% For LaTeX2e:
% \documentclass[<journame>]{agu2001}
% (e.g., \documentclass[jgrga]{agu2001})

% For LaTeX2.09:
% \documentstyle[<journame>]{agu2001}
% (e.g., \documentstyle[jgrga]{agu2001})

%%%%%%%%%%%%%%%%%%%%%%%%%%%%%%%%%%%%%%%%%%%%%%%%%%%%%%%%%%%%%%%%%%%%%%%%%%%%
%%  <journame> should be one of the following
%
% jgrga Journal of Geophysical Research
% gbc   Global Biogeochemical Cycles
% grl   Geophysical Research Letters
% pal   Paleoceanography
% ras   Radio Science
% rog   Reviews of Geophysics
% tec   Tectonics
% wrr   Water Resources Research
%
%%%%%%%%%%%%%%%%%%%%%%%%%%%%%%%%%%%%%%%%%%%%%%%%%%%%%%%%%%%%%%%%%%%%%%%%%%%%
%
% DRAFT MODE
%
% PLEASE USE THE DRAFT OPTION TO SUBMIT YOUR PAPERS
% The draft option produces double-spaced, full width output.
%
% See lines 80 & 155 if you need to include images in draft mode
% to make a pdf for submission.
%
% Substitute the journal you require for jgrga below.
% LaTeX2e:
\documentclass[draft,agums]{aguplus}
%
% LaTeX2.09:
%\documentstyle[draft,jgrga]{agu2001}
%

%%%%%%%%%%%%%%%%%%%%%%%%%%%%%%%%%%%%%%%%%%%%%%%%%%%%%%%%%%%%%%%%%%%%%%%%%%%%
%
%  GALLEY MODE
%
% PLEASE DO NOT USE THE GALLEY OPTION TO SUBMIT YOUR PAPERS
% The galley option produces single spaced, single column output
% It will display figures, but sends figures and tables to the end of the file
%
% LaTeX2e:
%\documentclass[galley,jgrga]{agu2001}
%
% LaTeX2.09:
%\documentstyle[galley,jgrga]{agu2001}
%

%%%%%%%%%%%%%%%%%%%%%%%%%%%%%%%%%%%%%%%%%%%%%%%%%%%%%%%%%%%%%%%%%%%%%%%%%%%%
%
%  PREPRINT MODE
%
% PLEASE DO NOT USE THE PREPRINT OPTION TO SUBMIT YOUR PAPERS
% This is the default option and it produces single spaced, double column output
% It does not always place figures and tables automatically.
%
% LaTeX2e:
%\documentclass[jgrga]{agu2001}
%
% LaTeX2.09:
%\documentstyle[jgrga]{agu2001}
%

%% ------------------------------------------------------------------------ %%
%
%  IMAGE DISPLAY
%
%% ------------------------------------------------------------------------ %%
%
% UNCOMMENT THE FOLLOWING LINE IF YOU NEED TO INCLUDE IMAGES IN EITHER
% DRAFT OR GALLEY MODE
%
% PLEASE NOTE: WE DO NOT WANT ANY GRAPHICS INCLUDED WHEN THE FINAL VERSION
% OF YOUR TEXT FILE IS SUBMITTED
%
% \usepackage{graphicx}

%% ------------------------------------------------------------------------ %%
%
%  PREAMBLE: RUNNING HEADS, COPYRIGHT, REC/REV/ACCEPTED
%  LINES, AND AUTHOR ADDRESSES
%
%% ------------------------------------------------------------------------ %%

\authorrunninghead{BALES ET AL.}
% Author names in capital letters,

\titlerunninghead{Short Title}
% Shorter version of title entered in capital letters

\authoraddr{R. C. Bales,
Department of Hydrology and Water Resources, University of
Arizona, Harshbarger Building 11, Tucson, AZ 85721, USA.
(roger@hwr.arizona.edu)}
%
% Author address will appear at end of article, may repeat
% this command for each author.

\begin{document}

%% ------------------------------------------------------------------------ %%
%
%  IMAGE DISPLAY IN DRAFT MODE
%
%% ------------------------------------------------------------------------ %%

% UNCOMMENT THE FOLLOWING CODE IF YOU NEED TO INCLUDE IMAGES IN DRAFT MODE
% TO MAKE A PDF FOR FIRST SUBMISSION.
%
%\setkeys{Gin}{draft=false}

% PLEASE NOTE: WE DO NOT WANT ANY GRAPHICS INCLUDED WHEN THE FINAL VERSION
% OF YOUR TEXT FILE IS SUBMITTED
% PLEASE COMMENT OUT ALL \includegraphics AND \figbox COMMANDS
% WHEN USING THE DRAFT MODE TO SUBMIT YOUR ACCEPTED ARTICLE
%
% For accepted papers
% (i) the graphics should not be included,
% (ii) figures and tables should be listed at the end of the file.
%


%% ------------------------------------------------------------------------ %%
%
%  TITLE
%
%% ------------------------------------------------------------------------ %%


\title{Title of Article}
%
% e.g., \title{Terrestrial Ring Current:
% Origin, Formation and Decay $\alpha\beta\Gamma\Delta$}

%% ------------------------------------------------------------------------ %%
%
%  AUTHORS AND AFFILIATIONS
%
%% ------------------------------------------------------------------------ %%

% Method 1 (for all journals, except Reviews of Geophysics, which
% should use method 3):
% For three or fewer author/affiliation blocks, use \author{} and \affil{}

\author{R. C. Bales}
\affil{Department of Hydrology and Water Resources,
University of Arizona, Tucson, Arizona, USA}

\author{E. Mosley-Thompson}
\affil{Department of Geography, Ohio State University,
Columbus, Ohio, USA}

\author{J. R. McConnell}
\affil{Desert Research Institute, Division of Hydrologic Sciences,
Reno, Nevada, USA}

% ---------------
% Method 2 (for all journals, except Reviews of Geophysics, which
% should use method 3): For more than three author/affiliation blocks,
% use \author{\altaffilmark{}} and \altaffiltext{}
% \altaffilmark will produce footnote; matching altaffiltext
% will appear at bottom of page. May use \\ to start a new line.

% \authors{R. C. Bales, \altaffilmark{1}
% E. Mosley-Thompson, \altaffilmark{2}
% R. Williams, \altaffilmark{3}
% and J. R. McConnell\altaffilmark{4}}

% \altaffiltext{1}
% {Department of Hydrology and Water Resources, University of Arizona,
% Tucson, Arizona, USA.}
%
% \altaffiltext{2}{Department of Geography, Ohio State University,
% Columbus, Ohio, USA.}
%
% \altaffiltext{3}{Department of Space Sciences, University of Michigan,
% Ann Arbor, Michigan, USA.}
%
% \altaffiltext{4}{Desert Research Institute, Division of Hydrologic Sciences,
% Reno, Nevada, USA.}

%---------------
% Method 3 (for Reviews of Geophysics only): Reviewauthors is a table with three
% columns. You must supply the ''&'' between each author/affiliation. If you have
% more than three authors, start a new table line with /cr

% e.g.,
% \begin{reviewauthors}
% R. C. Bales\\
% Department of Hydrology and\\ Water Resources\\
% University of Arizona\\
% Tucson, Arizona, USA
% &
% E. Mosley-Thompson\\
% Department of Geography\\
% Ohio State University\\
% Columbus, Ohio, USA
% &
% J. R. McConnell\\
% Desert Research Institute\\
% Reno, Nevada, USA
% \end{reviewauthors}

%% ------------------------------------------------------------------------ %%
%
%  ABSTRACT
%
%% ------------------------------------------------------------------------ %%

% Do NOT include any \begin...\end commands within
% the body of the abstract.

\begin{abstract}
(Type abstract here)
\end{abstract}

%% ------------------------------------------------------------------------ %%
%
%  TEXT
%
%% ------------------------------------------------------------------------ %%

% The body of the article must start with a \begin{article} command,
% and an \end{article} command must follow the references section.
% Otherwise, the text will not print at the appropriate column width.
%

\begin{article}

\section{Introduction}
(Type text here.)


%% ------------------------------------------------------------------------ %%
%
%  SECTION HEADS
%
%% ------------------------------------------------------------------------ %%

% Level 1 head

% Use the \section{} command to identify level 1 heads;
% type the appropriate head wording between the curly
% brackets, as shown below.
%
% Capitalize the first letter of each word (expect for
% prepositions, conjunctions, and articles that are
% three or fewer letters).
%
% Do not hyphenate level 1 heads. To break lines,
% type \protect\\ where you want the break to occur.
% AGU prefers the inverted triangle, breaking before
% prepositions, conjunctions, and articles, if possible.

\section{Level 1 Head: Introduction}
An example.

% ---------------
% Level 2 head

% Use the \subsection{} command to identify level 2 heads;
% type the appropriate head wording between the curly
% brackets, as shown below.
%
% Capitalize the first letter of each word (expect for
% prepositions, conjunctions, and articles that are
% three or fewer letters).
%
% Do not hyphenate level 1 heads. To break lines,
% type \protect\\ where you want the break to occur.
% AGU prefers the inverted triangle, breaking before
% prepositions, conjunctions, and articles, if possible.

\subsection{Level 2 Head}
An example.

% ---------------
% Level 3 head

% Use the \subsubsection{} command to identify level 3 heads;
% type the appropriate head wording between the curly
% brackets, as shown below.
%
% Capitalize only the first letter of the first word, acronyms,
% first letter of proper nouns, and first letter of first word
% after a colon.
%
% Hyphenation is permitted in level 3 heads, if needed.
%

\subsubsection{Level 3 Head} An example.  \subsubsubsection{Level 4 Head} An
example.

%% ------------------------------------------------------------------------ %%
%
%  EQUATIONS
%
%% ------------------------------------------------------------------------ %%

% Single-line equations are centered.
%
% Math coded inside display math mode $$...$$ will not be numbered e.g.:
% $$x^2=y^2 + z^2$$
%
% Math coded inside \begin{equation} and \end{equation} will
% be automatically numbered e.g.:
 \begin{equation}
 x^2=y^2 + z^2
 \end{equation}

% To create multiline equations, use the
% \begin{eqnarray} and \end{eqnarray} environment
% to break the equations into two or more lines, as
% demonstrated below.
% Use an \hspace{} command to insert horizontal space
% into your equation if necessary. Place an appropriate
% unit of measure between the curly braces, e.g.
% \hspace{1in}; you may have to experiment to achieve
% the correct amount of space.

% Example equation
\begin{eqnarray}
  x_{1} & = & (x - x_{0}) \cos \Theta \nonumber \\
        && + (y - y_{0}) \sin \Theta  \eqnum{5} \\
  y_{1} & = & -(x - x_{0}) \sin \Theta \nonumber \\
        && + (y - y_{0}) \cos \Theta. \nonumber
\end{eqnarray}

% Break each line at a sign of operation
% (+, -, etc.) if possible, with the sign of operation
% on the new line.
% Indent second and subsequent lines to align with
% the first character following the equal sign on the
% first line.
% If the left side of the equation is longer than one third
% of the line, subsequent lines should be indented the same
% amount as a text paragraph indent (0.38 cm, 0.15 inch).
% Set the equation number in parentheses flush right on the last
% line of the equation. If there is not enough room, set the
% number flush  right on a separate line immediately following
% the equation.




%% ------------------------------------------------------------------------ %%
%
%  EQUATION NUMBERING: COUNTER
%
%% ------------------------------------------------------------------------ %%

% You may change equation numbering by resetting
% the equation counter or by explicitly numbering
% an equation.
%
% To explicitly number an equation, type \eqnum{}
% (with the desired number between the brackets)
% after the \begin{equation} or \begin{eqnarray}
% command.  The \eqnum{} command will affect only
% the equation it appears with; LaTeX will number
% any equations appearing later in the manuscript
% according to the equation counter.
%
% To reset the equation counter, place the following
% command in front of your equation(s).
%
% Set the equation counter to 0 if the next
% number needed is 1 or set it to 7 if the
% next number needed is 8, etc.
%
% The \setcounter{equation} command does affect
% equations appearing later in the manuscript.
\setcounter{equation}{0}

% If you have a multiline equation that needs only
% one equation number, use a \nonumber command in
% front of the double backslashes (\\) as shown in
% equations above and below.
%
% ADDITIONAL EQUATION FEATURES
%
% To add letters after equation numbers, place your
% equation or eqnarray within a \begin{mathletters}
% and \end{mathletters} environment.  This environment
% can enclose several equations:

% \begin{mathletters}
% \begin{eqnarray}
% \gamma^\mu & = & \left(\begin{array}{cc} 0 &
% \sigma^\mu_+ \\
% \sigma^\mu_- & 0 \end{array} \right),
% \;\gamma = \left( \begin{array}{cc}
% \! \! -1 & \! \! 0 \\ \! \! 0 & \! \! 1
% \end{array} \right), \\ & & \nonumber \\
% \sigma^\mu_{\pm} & = & ({\bf 1} ,\pm \sigma),
% \end{eqnarray}
% giving
% \begin{eqnarray}
% \not\!a= \left(\begin{array}{cc}0 &
% (\not\!a)_+\\(\not\!a)_- &
% 0\end{array}\right),
% \;(\not\!a)_\pm=a_\mu\sigma^\mu_\pm\;,
% \end{eqnarray}
% \end{mathletters}

%% ------------------------------------------------------------------------ %%
%
%  IN-TEXT LISTS
%
%% ------------------------------------------------------------------------ %%

% Do not use bulleted lists; enumerated lists are okay.
% begin{enumerate}
% \item
% \item
% \item
% \end{enumerate}



%% ------------------------------------------------------------------------ %%
%
%  APPENDICES
%
%% ------------------------------------------------------------------------ %%
% \appendix resets counters and redefines section heads
% but doesn't print anything.
%
% \appendix
% \section{Your Title}
%
% will output the following:
% Appendix A: Your Title
%
%
% For those styles that allow an unnumbered or unlettered appendix, e.g.,
% Geophysical Research Letters, use
% \section*{Appendix}
% and it will print
% Appendix
%
% \section*{Appendix: Here is Appendix Title}
% will print
% Appendix: Here is Appendix Title

%% ------------------------------------------------------------------------ %%
%
%  NOTATION LIST
%
%% ------------------------------------------------------------------------ %%

% Notation will make the first column be in text mode.
% \begin{notation} starts a table that has two columns.

% \begin{notation}
% term & definition found here\\
% term & definition found here\\
% \end{notation}

% \notationwidth=<dimen> will make the first column in either environment
% be <dimen> wide, e.g., \notationwidth=3pc

% \setnotationwidth{<widest term here>} will let you set the width
%  of the first column to be a little wider than the term supplied;
%  e.g., \setnotionwidth{$\Delta H_{f,i}$}
%
% sample notation list
\begin{notation}
b &
type notation text describing the letter ``b.''\\
a$^{\prime}$      &
type notation text describing ``a$^{\prime}$.''  \\
f                &
type notation text describing the letter ``f.'' \\
O$_2$            &
type notation text describing ``O$_2$.''        \\
{\bf K}          &
type notation text describing ``{\bf K}.''       \\
g                &
type notation text describing the letter ``g.'' \\
\end{notation}

%% ------------------------------------------------------------------------ %%
%
%  GLOSSARY (for Reviews of Geophysics and Geochemistry,
%  Geophysics, Geosystems only)
%
%% ------------------------------------------------------------------------ %%

% \section*{Glossary}
% \paragraph{term}
% Text describing this term...

% \paragraph{term}
% Text describing this term...

%% ------------------------------------------------------------------------ %%
%
%  ACKNOWLEDGMENTS
%
%% ------------------------------------------------------------------------ %%

\begin{acknowledgments}
(text here)
\end{acknowledgments}

%% ------------------------------------------------------------------------ %%
%
%  REFERENCE LIST AND TEXT CITATIONS
%
%% ------------------------------------------------------------------------ %%

\begin{thebibliography}{}

\bibitem{}
Anklin et al., Title of article, {\it Journal Abbrev., 100,}
1--20, 1998.
%
%In text, write the author name and date. 
%
%If you use BiBTeX
%1. Run LaTeX on your LaTeX file.
%
%2. Run BiBTeX on your LaTeX file.
%
%3. Open the new .bbl file containing the reference list and 
%copy all the contents into your LaTeX file after the 
%acknowledgments section;
%
%4. Comment out the old \bibliographystyle and \bibliography commands.
%
%5. Run LaTeX on your new file before submitting.

%Failure to follow these instructions will require manual 
%intervention through hard keying of information, 
%which can introduce errors.

\end{thebibliography}

%% ------------------------------------------------------------------------ %%
%
%  END ARTICLE (1/2)
%
%% ------------------------------------------------------------------------ %%
% PLEASE PLACE END ARTICLE AND NEW PAGE COMMANDS HERE FOR DRAFT MODE
% FOR GALLEY MODE REMEMBER TO (1) COMMENT OUT THESE LINES AND
% (2) PLACE AN END ARTICLE COMMAND AFTER THE FIGURES AND TABLES INSTEAD
\end{article}
\newpage



%% ------------------------------------------------------------------------ %%
%
% FIGURES (see end of examples for further instructions)
%
% ---------------
% Single column figure example
%

\begin{figure}
% \noindent\includegraphics[width=20pc]{samplefigure.eps}
 \caption{Caption text here}
\end{figure}



%% ------------------------------------------------------------------------ %%
%
% TABLES (see end of examples for further instructions)
%
% ---------------
% Single column table example
%
%\clearpage
\begin{table}
\caption{Summary of Correlations Between Ice Cores and
Indices\tablenotemark{a}}
\begin{flushleft}
\begin{tabular}{lcccc}
\tableline
\multicolumn{1}{c}{Site}&Time&12-Month&Pearson's&Spearman \\
&Span&Period&$R$&Rank Order \\
\tableline
\multicolumn{5}{c}{{\it First Example Italic Centered Heading}} \\
GITS\tablenotemark{b} & 1865--1995&Feb.--Jan.&-0.316&-0.298 \\
Camp Century & 1865--1974&July--June&-0.320&-0.298 \\
\multicolumn{5}{c}{{\it Second Example}} \\
Nasa-U & 1865--1994&Sept.--Aug.&-0.353&-0.342 \\
Milcent & 1865--1966&June--May&-0.410&-0.494 \\
\tableline
\end{tabular}
\end{flushleft}
\tablenotetext{a}{This is an example of the tablenotetext
command.} \tablenotetext{b}{Here is a second example.}
\end{table}


% ---------------
% Table 2 - double column table example:
%
%\clearpage
\begin{table*}
\def\ph{\phantom{age }}
\def\pph{\phantom{th}}
\newbox\dothis
\setbox\dothis=\vbox to0pt{\vskip-1pt\hsize=11.5pc\centering
Error\vss} \caption{Please Note That This Double-Column Table Does
Not Display Properly in Draft Mode\tablenotemark{a}}
\begin{tabular*}{\textwidth}{@{\extracolsep{\fill}}lccrrrcrrr}
\hline &&\multicolumn{4}{c}{\vrule height 12pt width 0pt Panel A
Regression A}&\multicolumn{4}{c}{Panel B Regression B}\cr
\cline{3-6}\cline{7-10}\cr
%&&\multicolumn{4}{c}{\hrulefill}&\multicolumn{4}{c}{\hrulefill}\cr
\noalign{\vskip-4pt} &&&\multicolumn{3}{c}{\copy\dothis}&
&\multicolumn{3}{c}{\copy\dothis}\cr \cr \noalign{\vskip-6pt}
\cline{4-6}\cline{8-{10}}\cr \noalign{\vskip-6pt}
&Actual&Predicted&&\multicolumn{2}{c}{Cumulative}
&\multicolumn{1}{c}{\ Predicted}
&&\multicolumn{2}{c}{Cumulative}\cr \cline{5-6}\cline{9-{10}}\cr
\noalign{\vskip-9pt}
&\multicolumn{1}{c}{M2}&\multicolumn{1}{c}{M2}
&\multicolumn{1}{c}{Level\hbox to-12pt{}}&&&\multicolumn{1}{c}{M2}&%
\multicolumn{1}{c}{Level}&\cr
%%
Year&\multicolumn{1}{c}{Growth} &\multicolumn{1}{c}{Growth}
&\multicolumn{1}{c}{Growth}&
\multicolumn{1}{c}{Billns\hbox to-12pt{}}&Percentage&%
\multicolumn{1}{c}{Growth} &\multicolumn{1}{c}{Growth}&
\multicolumn{1}{c}{Billns\hbox to -12pt{}}&Percentage\cr
\noalign{\vskip3pt} \hline \noalign{\vskip3pt} 1990Q4&4.0& 6.4&
$-$2.3\ &$-$71 & 2.2\ph& 6.5&$-$2.4\pph&$-$80&2.4\ph  \cr
1991Q4&3.0& 3.6& $-$0.5\ &$-$91 & 2.7\ph &
3.3&$-$0.3\pph&$-$92&2.7\ph  \cr 1992Q4&1.8& 6.4& $-$4.5\ &$-$257
& 7.5\ph  & 5.9&$-$4.0\pph&$-$239&6.9\ph  \cr 1993Q4&1.4& 4.8&
$-$3.4\ &$-$392 & 11.2\ph  & 5.0&$-$3.6\pph&$-$381&10.9\ph  \cr
1994Q4&0.6& 3.0& $-$2.4\ &$-$489 & 13.9\ph  &
2.6&$-$2.0\pph&$-$464&13.2\ph  \cr 1995Q4&3.8& 3.5& 0.3\ &$-$495 &
13.6\ph  & 4.2&$-$0.4\pph&$-$500&13.7\ph  \cr 1996Q4&4.5& 3.9&
0.5\ &$-$495 & 13.0\ph  & 4.0&$-$0.4\pph&$-$505&13.3\ph\cr
\noalign{\vskip3pt} \multicolumn{3}{l}{Mean Error (1990\/--1996)}&
\multicolumn{4}{l}{\phantom{$-$}$-$1.78}&
\multicolumn{3}{l}{\phantom{$.$}$-$1.78}\cr \multicolumn{3}{l}{\it
{\rm RMSE}}& \multicolumn{4}{l}{\phantom{$--$}2.52}&
\multicolumn{3}{l}{\phantom{$-.$}2.40}\cr \hline
\end{tabular*}
\tablenotetext{a}{Please note that this double-column table does
not display properly in draft mode. This is an example of the
tablenote command. The predicted values are generated using the
regressions reported in Table 1. Regressions are estimated from
1960Q4 and dynamically simulated from 1990Q1 to 1966Q4. RMSE is
the root-mean-square error, which is of particular interest in
this context.}
\end{table*}

%% ------------------------------------------------------------------------ %%
%
%  END ARTICLE (2/2)
%
%% ------------------------------------------------------------------------ %%
% PLEASE PLACE END ARTICLE COMMAND HERE FOR GALLEY MODE
% FOR DRAFT MODE PLEASE REMEMBER TO (1) COMMENT OUT THIS LINE AND
% (2) PLACE AN END ARTICLE COMMAND AFTER THE BIBLIOGRAPHY INSTEAD
%\end{article}
%

%% ------------------------------------------------------------------------ %%
%
%  FIGURE, PLATE, AND TABLE NUMBERING: COUNTERS
%
%% ------------------------------------------------------------------------ %%
% To set counters explicitly:
% eqnum =  for current equation number, will not affect following numbers
%
% May need to do this for figures or tables near the appendix:
% \setfigurenum{<number>}, following figures will follow this number
% \settablenum{<number>}, following tables will follow this number

% ------------------------------------------------------------------------ %%
%
%  FIGURE, PLATE, AND TABLES
%
% If you need to display your figures in draft mode for SUBMISSION,
% place the code \setkeys{Gin}{draft=false} immediately after \begin{document}
% and use the \includegrahics command (see below) to call the eps graphics file.
%
% PLEASE COMMENT OUT ALL \includegraphics AND \figbox COMMANDS
% WHEN USING THE DRAFT MODE TO SUBMIT YOUR ACCEPTED ARTICLE
%
%% ------------------------------------------------------------------------ %%
%
%
% Single column figure/table, including eps graphics
%
% \begin{figure}
% \noindent\includegraphics[width=20pc]{samplefigure.eps}
% \caption{Caption text here}
% \end{figure}
% \end{document}
%
% \begin{table}
% \caption{}
% \end{table}
%
% ---------------
% Double column figure/table
%
% \begin{figure*}
% \noindent\includegraphics[width=39pc]{samplefigure.eps}
% \caption{Caption text here}
% \end{figure*}
%
% \begin{table*}
% \caption{Caption text here}
% \end{table*}
%


% ---------------
% Landscape (broadside) figure/table
% LANDSCAPE FIGURES AND TABLES DO NOT WORK IN DRAFT MODE
%
% ONE-COLUMN landscape figure and table
%
% \begin{landscapefigure}
% \includegraphics[height=.75\mycolumnwidth,width=42pc]{samplefigure.eps}
% \caption{Caption text here}
% \end{landscapefigure}
%
% \begin{landscapetable}
% \caption{Caption text here}
% \begin{tabular*}{\hsize}{@{\extracolsep{\fill}}lcccc}
% \tableline
% ....
% \tableline\\
% \multicolumn5l{(a) Algorithms from Numerical Recipes}\\
% \end{tabular*}
% \tablenotetext{}{}
% \tablecomments{}
% \end{landscapetable}
%
% FULL-PAGE landscape figures and tables
%
% \begin{figure*}[p]
% \begin{landscapefigure*}
% illustration here
% \caption{caption here}
% \end{landscapefigure*}
% \end{figure*}
%
% \begin{table}[p]
% \begin{landscapetable*}
% \caption{}
% \begin{tabular*}{\textheight}{@{\extracolsep{\fill}}lccrrrcrrr}
% ....
% \end{tabular*}
% \begin{tablenotes}
% ...
% \end{tablenotes}
% \end{landscapetable*}
% \end{table}
%
%% ------------------------------------------------------------------------ %%
%
%  ALTERNATE CAPTIONS (for Reviews of Geophysics only)
%
% PLEASE NOTE THAT THESE COMMANDS DO NOT WORK IN DRAFT MODE
%
%% ------------------------------------------------------------------------ %%
%
% \sidecaption{}{}
% First argument (set of brackets) is for illustrations, will take 0.7
% of the page width; second argument is for the caption, which will
% center vertically and appear to the right of the illustration.
%
% For two small illustrations side by side:
% \begin{figure*}
% \sidebyside{}{}
% \end{figure*}
%
% \begin{figure*}
% \sidebyside{illustration \caption{}}{ illustration \caption{}}
% \end{figure*}
%

\end{document}


